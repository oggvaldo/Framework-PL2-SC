\chapter{A importância da governança TI para o cenário profissional}
\section{A importância da governança TI para o cenário profissional}
É inegável que temos desde os primórdios da mistura entre negócios e tecnologia da informação uma certa preocupação em analisar e quantificar como não somente a adequação do sistema tem ocorrido, mas também se os recursos humanos o qual usufrirão de tais sistemas estão fazendo-o corretamente. Verificar se tais investimentos de finanças e esforços acadêmicos estão retornando resultados é algo essencial, porém mesmo com uma abundante quantidade de estudos que vemos saindo dia após dia, ainda pairam dúvidas se de fato temos como quantificar e qualificar este esforço. Somente assim podemos ter dados de tais métricas e como os problemas possam ser mapeados e corrigidos, qualidades possam ser aprimoradas e verificar "pontos vazios" para a criação ou inserção de novas técnicas.  Porém, como o autor abaixo explica:
\begin{quotation}

    "Enquanto grandes investimentos vêm ocorrendo na área de informática, muito pouco se sabe sobre seus efeitos nas organizações, especialmente porque o relacionamento entre a TI e o desempenho organizacional tem se mostrado bastante complexo e multifacetado, dificultando a identificação e a avaliação do impacto financeiro destes investimentos. Willcocks e Lester (apud LIN; PERVAN, 2001\cite{LIN}) apontam três justificativas pelas quais a identificação e a avaliação do impacto organizacional da TI são prejudicadas. São elas: (a) muitos executivos acreditam que não existe uma solução viável para esse problema, pois por razões competitivas percebem que não podem deixar de investir em TI, mesmo que não encontrem uma justificativa economicamente plausível; além disso, (b) como a infra-estrutura de TI se torna uma parte inseparável dos processos e da estrutura da organização, torna-se difícil separar o impacto proporcionado pela TI das demais atividades da organização; e (c) a existência de uma lacuna na identificação e compreensão dos custos, benefícios e riscos envolvidos nas diferentes tecnologias adotadas que dificulta a visualização do retorno de cada uma delas."(LUNARDI; 2008)\cite{ESTUDO-EMPIRICO}


\end{quotation}    
    
É explícito que o investimento somente na área de tecnologia não é a solução definitiva de tais problemas, visto que mesmo com a proposta de tentar facilitar processos e procedimentos ainda é fortemente dependente de humanos responśaveis pela execução destes processos. Verificar como se ocorrerá a gestão de mudança, tanto pela parte comportamental quanto pela parte da cultura da empresa, não somente para que o uso da ferramenta possa ser eficiente, porém principalmente para que o investimento realizado em tais ferramentas de suporte façam sentido ao gasto demandado por tal problema. O uso das ferramentas de TI por si só não é capaz sozinha de aumentar os ganhos da produtividade, seu correto uso sim. Não somente pela organização, como um todo, mas principalmente de forma nuclear, no funcionário que realiza o trabalho considerado como "chão de fábrica", onde ele será o operador final de tal ferramenta. Este funcionário, que é o motivo da empresa prosperar e manter seu nome como um todo, se não for corretamente condicionado ao uso de tal ferramenta, pode ser o início de uma corrente extremamente desastrosa, em especial no aspecto financeiro. E é neste aspecto onde mora o diferencial entre empresas e empresas, onde com o uso da mesma ferramenta de suporte, ela é capaz de obter extremos ganhos ou perdas, determinando o sucesso não somente comercial mas também até mesmo o nome e tradição da empresa, caso esta já possua uma.

A habilidade - ou a ausência de tal - tem sido o principal determinante para o posicionamento estratégico de diversas empresas, seja para o sucesso ou para o fracasso. (apud WEILL; OLSON\cite{WEILL}) Tal conceito é chamado de {\bf efetividade de conversão} segundo os autores Peter Weill e Margarethe Olson, em março de 1989 (mostrando a nós como tal conceito tem mais tempo do que imaginávamos). A partir deste ponto de vista, diversos pesquisadores ao redor do globo tem pesquisado e procurado propor modelos teóricos para analisar como os investimentos, que tanto foram citados nas páginas anteriores, podem criar valor para os negócios, aumentar a produtividade ou, até mesmo, ampliar o desempenho organizacional. A partir desta ótica, é possível verificação entre o sucesso e o fracasso no valor gerado pela TI para o seu foco de serviço, determinando até que ponto esta área funcionou como uma catalisadora para seus negócios e, assim, trazer capacidade de organização de gerência e potencialilzação de investimentos.

\section{Governança em TI, de onde veio?}
Por curiosidade, o termo {\bf Governança em TI} surge como uma tentativa de garantir a agregação de valor aos negócios da organização após investimentos realizados em tecnologia (apud DE HAES; VAN GREMBERGEN, 2005)\cite{DEHAES2005}. Desta forma, verificamos que a governança de TI gera um efeito direto sobre a gestão da TI, visto onde é através dela que um conjunto de regras são criadas, definidas, postas em ação e avaliadas para governar toda a função da TI na organização (apud VERHOEF, 2007 \cite{VERHOEFF}).

Tal metodologia começou a ser largamente adotada após no final dos anos 90, com a quebra de grandes empresas americanas, decorrente de fraudes em seus relatórios financeiros, onde em um cenário em que a governança corporativa e responsabilidade fiscal passaram a ter um grande interesse no meio empresarial, apoiado por um meio de justificar e, principalmente, dar mais efetividade nos investimentos realizados nesta área.

Usando as palavras do autor Guilherme Lerch:
\begin{quotation}
\textit{"A governança de TI, propriamente dita, envolve a aplicação de princípios de
Governança Corporativa para dirigir e controlar a TI de forma estratégica, preocupando-se
exclusivamente com dois assuntos-chave: o valor que a TI proporciona à organização, e o
controle e a diminuição dos riscos relacionados à TI (ITGI, 2003\cite{ITGI}; PETERSON, 2004b\cite{PETERSON}; HARDY, 2006\cite{HARDY}). O primeiro assunto é direcionado pelo alinhamento estratégico entre os negócios e a tecnologia, enquanto que o segundo é direcionado pela definição dos responsáveis na organização pelas decisões envolvendo os assuntos ligados à TI. Para que isso ocorra, é necessário que os recursos tecnológicos da organização sejam adequados e que o seu desempenho seja constantemente mensurado (ITGI, 2003\cite{ITGI})."}
\end{quotation}

Ora, é possível notarmos aqui que a governança da TI vai muito além do que a gestão da TI, vemos como todas as questões da organização, sejam quaisfor, estão sendo suportadas e relacionadas diretamente pela tecnologia, desde a definição dos direitos e responsabilidades sobre a área, passando pelo estudo e aprovação dos de investimentos tecnológicos para as melhorias do negócio, pelo monitoramento e manutenção da TI já presente no local, até chegar na avaliação do valor dado pelo departamento ao negócio. Daí, o fator da \textbf{efetividade de conversão da TI} neste caso não é ligado apenas a delegação do uso da tecnologia pela empresa, porém juntamente, as decisões que são antecedidas da sua aquisição, bem como o valor que o uso desta ferramenta pode impactar a organização.

Logo, não é incomum vermos que empresas que dispõem de tal política como algo inerente à cultura e organização possuem uma vantagem, tanto financeira quanto organizacional, para estarem se aproximando de uma posição privilegiada no mercado, se não a posição de destaque a depender do mercado o qual atua e de sua gerência, visto que a aplicação dos mesmos {\it frameworks} podem gerar distintos resultados a depender da empresa e sua cultura, organização e até mesmo processo de implantação.

\section{Cultura organizacional e gestão de mudança}

Para entendermos um pouco melhor, vamos explorar como e o que é a cultura organizacional de uma empresa. Segundo o texto retirado do site portal-administração, o conceito é:

\begin{quotation}
...a cultura organizacional é um conjunto de hábitos, crenças e valores, que por sua vez, são estabelecidos através de normas, princípios, atitudes e perspectivas compartilhadas pelos colaboradores de uma empresa. Basicamente, ela constitui o modo de pensar e agir da companhia (sendo um dos principais fatores que diferencia uma empresa das demais). Podemos dizer também que a essência da cultura de uma empresa é a própria maneira como a mesma realiza seus negócios, lida com seus clientes e colaboradores e o grau de lealdade que eles transmitem para com a organização.(PORTAL DA ADMINISTRAÇÃO, 2014)\cite{PORTALADM}
\end{quotation}

O esclarecimento do conceito acima nos ajuda a começar a entender como e por onde podemos começar o desenvolvimento de uma política de governança em TI. Para tanto, façamos o seguinte exercício. Vamos imaginar uma grande empresa onde, tradicionalmente, boa parte de seus processos de negócios eram feitos através de processos manuais. Desde a aferição de balanços até a contagem de estoques, passando pelo registro de folhas de ponto. Nós, como estudantes da área de tecnologia logo imaginamos porque não há um sistema que pode gerenciar tudo isto (afinal, tais sistemas existem já faz muitos anos). 

Porém após realizarmos que antes da governança, devemos nos lembrar da gestão, já começamos a imaginar onde podem ser os possíveis gargalos. O primeiro já identificado por inferência ao longo do texto é de como esta gestão de mudança deverá agir em conformidade com a cultura organizacional da empresa. Por mais que a empresa e seus colaboradores compartilhem de um determinado comportamento, uma gestão de mudança sempre pode ser fortemente afetada caso não esteja em harmonia com o momento da empresa, preparação dos colaboradores, preparação do ambiente, capacidade financeira para tal e, principalmente, se a empresa se encontra no momento certo para tal coisa. 

E para corroborar como o momento certo pode influênciar, de acordo com o professor Reinaldo Lucas:
\begin{quotation}
"...tudo na vida funciona dentro de um ciclo, que é chamado ciclo da evolução dos organismos. Este ciclo tem cinco etapas: formação, tumulto, normalidade, desempenho e acomodação. Qual é o tempo deste ciclo? É muito relativo. Dizem que na vida dos indivíduos é em torno de sete anos. Há estudos que mostram que as empresas no Brasil morrem depois do terceiro ciclo. Então, é importante percebermos qual é o momento em que a empresa começou a se acomodar e aí fazer uma mudança. Esta mudança normalmente tem que envolver a sua estratégia, a sua estrutura, os seus processos e as pessoas. Então, qual é o momento ideal para mudar? É o momento em que a empresa começou a acomodar, mas também não se pode ter um processo constante de mudança porque a mudança constante destrói a empresa. É preciso ter sempre períodos de evolução e períodos de revolução. A medição acontece muito em função do tipo de negócio. Se pegarmos uma indústria de telefone celular, essa mudança é muito acelerada pela tecnologia. Se pegarmos uma mineradora, o tempo entre uma mudança e outra é muito maior. Então, depende muito do segmento de negócio."\cite{FDC}
\end{quotation}

Porém, para melhor compreendermos tal questionamento, o que é uma mudança dentro deste contexto? Segundo Chiavenato:

\begin{quotation}
Mudança é a transição de uma situação para outra diferente ou passagem de um estado
para outro diferente. Mudança implica ruptura, transformação, perturbação,
interrupção. O mundo atual se caracteriza por um ambiente dinâmico em constante
mudança e que exige das organizações uma elevada capacidade de adaptação, como
condição básica de sobrevivência. Adaptação, renovação e revitalização significam
mudança.\cite{CHIAVENATO}
\end{quotation}

É nestas horas que paramos para pensar e perceber como a implantação de uma política de governança pode ser mais difícil do que se imagina. Ter que pensar em todos os tópicos acima citados, além de realizar a gestão pré, durante e pós implementação de algo novo são processos penosos. Porém, ainda assim, temos casos de como uma correta gestão de mudança, aliada a uma forte política de governança em TI pode maximizar ganhos financeiros de empresas.

Um dos inúmeros cases que temos disponíveis para consulta foi realizado por dois professores, em que descrevem a implantação das métricas e políticas que vieram em conjunto com as práticas de governança em TI. Este cenário de estudo foi em uma empresa de telecom, a qual não tem seu nome divulgado por motivo não explicado no artigo apresentado. Esta empresa só veio adotar novas políticas de governança após uma mudança no setor de telecom após a criação da lei 9.472 (BRASIL, 1997)\cite{9472}.

O ponto levantado pelos pesquisadores Sortica e Graeml após a pesquisa pode ser lido de forma resumida abaixo:
\begin{quotation}
"O objetivo do estudo que motivou este artigo foi compreender a forma de utilização dos
critérios de efetividade da governança de TI (tecnologia da informação) na implementação de
estratégias por uma empresa fornecedora de serviços para o setor de telecomunicações. Para tanto, foram verificadas a existência e a utilização de critérios de efetividade tático-operacional, definidos nos modelos de governança tecnológica. 

A seguir, foi verificada a existência e a utilização
de critérios de efetividade estratégica, conforme apresentados na literatura. O estudo de caso
envolveu a realização de entrevistas semi-estruturadas, a partir das quais foram obtidos os dados para análise, em adição a dados documentais disponibilizados pela empresa. Observou-se que todos os critérios de efetividade tático-operacionais previstos nas categorias estudadas foram implementados na organização, embora se tenha verificado que algumas categorias de critérios de efetividade estratégica não estejam presentes ou não estejam completamente implementadas.

Detectou-se uma sobrevalorização de aspectos tático-operacionais da operação, quando contrastados com os aspectos estratégicos do negócio, o que é comum em empresas que têm processos produtivos complexos, que precisam ser bem gerenciados para que se garanta a qualidade do produto ou serviço oferecido ao mercado. \cite{Sortica}"
\end{quotation}

Mostrando o resultado com um ponto deste assunto, e um dos mais importantes, segue citação:
\begin{quotation}
"A governança tecnológica, como metodologia e parte fundamental da governança corporativa,
traz um conjunto de benefícios técnicos e operacionais à organização analisada e ao seu relacionamento
com as operadoras de telecomunicações (seus clientes). As prerrogativas da gestão,
como lucratividade e fidelidade aos investidores são dependentes de princípios estabelecidos para
a QoS (quality of service), como disponibilidade dos sistemas de informação, dependente da
criticidade do processo; interdependência dos processos; segurança da informação; e desempenho
dos serviços de TI por meio dos SLA."  \cite{Sortica}
\end{quotation}

Até aqui, temos um cenário ideal, um cenário onde o que se compõe são as metodologias, o maquinário, os {\it frameworks} e demais aspectos que servem de ferramentas, mas e como o material humano entra nesta área? Como será que a parte que pode colocar um negócio tanto em posição de destaque quanto levar à falência empresas tem agido perante tal cenário, que mesmo sendo estudado há mais de 20 anos, com testes e novas tecnologias surgindo a todo o momento para facilitar a adaptação da empresa e até mesmo do material humano envolvido, tem sido trabalhada? Será que o processo da implantação está passando somente por um processo puramente administrativo? Será que o processo de transmitir o que a empresa deseja a seu colaborador tem sido feito de uma forma efetiva? 

\section{Como a cultura organizacional e a governança em TI podem se relacionar}

Durante a pesquisa sobre como a cultura organizacional e a governança podem se relacionar, nos deparamos com uma pesquisa de três professores, onde abrem o seu artigo com a tese que é possível sim basear um modelo de governança de TI relacionando com o framework de Deter et al's (2000). Para tanto, primeiro foi necessário ligar seus métodos ao cenário de governança em TI. Como já vimos em partes como funciona a cultura organizacional de uma empresa, vamos listar abaixo os oito pontos que este framework possui.\cite{exploringit}

\subsection{Modelo de cultura organizacional Detert et al (2000)}
    \subsubsection{2.4.1.1. A base da verdade e da racionalidade}
    Se foca no grau onde cada colaborador acredita que algo é ou não real e como a verdade pode ser descoberta. Esta dimensão pode afetar o grau onde as pessoas podem adotar ideais tanto normativos quanto pragmáticos. Em outras palavras, a extrensão onde cada organização observa a verdade através de um estudo sistêmico e científico usando os dados brutos (uso de dados para tomadas de decisão) ou através de experiências pessoais e intuitivas {\bf [Dados brutos x Experiência pessoal para a tomada de decisões]}
    
    \subsubsection{2.4.1.2. A natureza do tempo e o tempo corrido}
    O conceito de tempo em uma organização tem exposto em termos de que qualquer organização tem adotado planos de longo-prazo, planos estratégicos e com um fim bem definido, ou foca primariamente no "aqui e agora", reagindo em um curto tempo, praticamente em tempo real. Expllicando, a extensão que cada organização foca em termos a longo ou curto prazo. {\bf [Longo prazo x Curto prazo]}
    
    \subsubsection{2.4.1.3. Motivação}
    Acreditar no que humanos são motivados é algo fundamental. Internamente, a motivação organizacional é um princípio fundamental da gerência. A identificação de como os colaboradores estão motivados, se eles são motivados por uma força interna ou externa é importante. Além do mais, como gerentes com mais tempo de casa podem acreditar como uma tecnologia pode determinar a sua própria importância na organização. Em miudos, a extensão de como a TI pode ser vista como um custo ou como pode se tornar um recurso que entregará valor para a organização. {\bf [Custo x Valor]}
    
    \subsubsection{2.4.1.4 Orientações para a mudança/inovações}
    Estabilidade e mudanças estão intimamente ligadas a motivação. Alguns indivíduos estão abertos para mudanças (pessoas que gostam de lidar com maiores riscos), enquanto outros precisam de uma maior necessidade de estabilidade (aversos a riscos). Isto pode também ser aplicado a organizações como um todo. Organizações mais dispostas a assumirem riscos são ditas por serem organizações que vão sempre atrás de inovações, em uma constante e contínua melhora, enquanto organizações mais conservadoras tendem a ser menos inovadoras, com uma pequena vontade de ocasionar mudanças. essencialmente, este é o limiar onde as organizações tem a propensão em se manterem em um nível estável o suficiente de performance que é "bom o bastante" ou se elas sempre estão procurando se melhorarem através de inovações e mudanças. {\bf [Estabilidade x Mudanças]}
    
    \subsubsection{2.4.1.5 Orientação ao trabalho, tarefa ou processo}
    A centralidade de um trabalho na vida humana e o balanço entre trabalho e produção no trabalho e vida social. Alguns indivíduos veem trabalho como um fim nele próprio e estão preocupados em chegar ao ponto de chegada e produtividade apenas. Outros indivíduos enxergam o trabalho como um meio, como uma ferramenta para alcançar uma vida confortável e desenvolvimento de relações sociais. Alguns problemas como a responsabilidade de como os colaboradores se sentem na posição que se encontram e como eles estão sendo educados em termos de suas responsabilidades e papéis são importantes neste ponto. Em resumo, a extensão de como cada indivíduo nas oraganizações focam no seu trabalho como um fim (a procura de resultados) ou como eles focam no processo onde o trabalho é um meio para alcançar a outros fins. {\bf [Processos x Resultados]}

    \subsubsection{2.4.1.6 Isolamento contra Cooperação/Colaboração}
    Foca em como os colaboradores podem trabalhar, sejam sozinhos ou de forma cooperativa. Está ligada em relacionar a natureza das relações humanas e como o seu trabalho pode ser concluido de forma mais efetiva e eficiente. Em algumas organizações, a maior parte do trabalho são concluidas por indivíduos, e a colaboração é sempre vista como uma vioalção à autonomia. Outras organizações enxergam de forma positiva o trabalho em grupo, sempre organizando suas equipes conforme o trabalho. De outra forma, a extensão em como organizações encorajam a colaboração sobre os indivíduos sobre o trabalho em equipe ou se encorajam o trabalho individual sobre o trabalho em equipe. {\bf [Isolamento x Cooperação]}
    
    \subsubsection{2.4.1.7 Controle, coordenação e responsabilidade}
    Organizações variam no nível onde o comando é concentrado ou dividido. Onde a há uma espécie de controle da firma, existem regras formais e procedmientos que são seguidos por alguns, um guia de comportamento da maioria. Onde ha menos controle, existe flexibilidade e autonomia de seus colaboradores, com poucas regras ou procedimentos formais e a tomada de decisão geralmente é em equipe. Ou seja, a extensão onde cada organização tem a estrutura da tomada de decisão estruturada em um único ponto focal contra uma estrutura tomada de decisão montada através da divisão de tarefas e, por consequente, divisão de tomadas pela organização. {\bf [Centralização x Autonomia nas tomadas de decisões]}
    
    \subsubsection{2.4.1.8 Foco e orientação - interna e externa}
    A natureza de como a relação entre uma organização e seu ambiente e quando ou não uma organização assume seu controle, ou por quem é controlado, seu ambiente alheio. Uma organização pode ter uma orientação mais interna (focada em pessoas e processos dentro da organização) ou externa (focada em sua constituição, usuários, competidores e o ambiente como um todo), ou ter uma combinação de ambos. A extensão para qual tipo de desenvolvimento organização é focada passa ou pelos processos internos de melhoramentos ou como o seu investidor deseja. {\bf [Interno x Externo]}
    
Após um estudo realizado por De Haes, Grembergen e Rowlands\cite{exploringit}, primeiro eles fizeram uma identificação dos comportamentos relacionados aos valores de uma cultura que cruzam com a implantação de uma governança em TI de forma a teorizar sobre como a visão da cultura podem facilitar ou impedir a implantação da política de governança em TI.

Os passos tomados pelos autores para o estabelecimento foram o seguinte, conforme retirado diretamente do artigo deles.

{\it To apply Detert et al’s, (2000) general cultural dimensions framework to our specific initiative, our first step was to scan the ITG literature to determine what normative dimensions have been used to define the ideal culture of an ITG organisation. The second step of our approach was to focus on translating the eight dimensions originally proposed by Detert and to link them to the cultural values underlying ITG. For instance we re-interpreted dimension #1 truth and rationality ( the degree to which employees believe something is real or not real and on how the truth is discovered), as the extent to which organisations seek truth through systemic, scientific study using hard data (the use of data for decision making) or through personal experience and intuition.

We interpreted dimension #2 the nature of time and time horizon in terms of whether the organisation adopt long-term planning, strategic planning and goal setting, or focus primarily on the here and now, reacting on a short time horizon. We interpreted dimension #3 motivation as to how senior management believe in a technology’s worth will determine its role in an organisation. We see motivation as the extent to which an organisation believes that IT is a cost, or alternatively that IT is seen as an asset and can deliver value for an organisation; and so on. Likewise we interpreted dimension #8 orientation in terms of whether an organisation assumes that it controls, or is controlled by, its external environment.

The third step, in order to validate the 8-dimensional framework of an ITG culture framework, we sought feedback on our initial summary table from a focus group (Tremblay et al, 2010) of eight comprised of practitioners and consultants in ITG who attended an advanced seminar on Cobit5 at the Antwerp Management School in 2013. The authors provided the focus group with draft working definitions of an ITG culture based on the work of Detert et al (2000), and our translation of what each value meant in an ITG context (see Table 1). Based on recommendations of Van de Ven and Delbecq (1972) the focus group members were then asked to articulate aspects of ITG culture that they believed to be critical. The ITG practitioners were informed that their input would assist in validating the conceptualisation of an ITG culture. The practitioners then read each culture value and made written changes or confirmations. This procedure rendered minor corrections and enhancements to our initial conceptualisation of the eight dimensions of ITG culture and are presented in table 2. We suggest that the outcome of the expert focus group’s input significantly adds to our understanding of the ITG culture phenomenon.}

O resultado, já traduzido, foi a tabela abaixo:

\begin{center}

    \begin{tabular}{|p{5cm}|p{11cm}|}
        \hline
        {\bf Dimensão da cultura organizacional (Detert et. al, 2000)} & {\bf Valor segundo a governança em TI}  \\ \hline
        1. A base da verdade e da racionalidade & Tomada de decisão relativa ao gasto de capital em novos sistemas e arquiteturas de TI que devem satisfazer em relação à quantidade de informações factuais disponíveis, um processo transparente, levando em conta a disponibilidade de riscos da empresa. \\ \hline
        
        2. A natureza do tempo e o tempo corrido & Melhoramento para os processos de TI, um melhoramento na maturidade do processo e alinhamento da TI com o negócio requerem uma orientação a longo prazo, onde uma aproximação flexível a nível tático e um requerimento a curto prazo. \\ 
        \hline
        
        3. Motivação & Organizações necessitam enxergar a TI não como um custo, mas sim focarem no valor do negócio em co-criação com a TI. \\
        \hline
        
        4. Estabilidade contra Mudança/Inovação & Os melhoramentos dos processos de TI são contínuos, e podem ser melhorados com o uso de recursos. Organizações devem sempre olhar em otimizar e mudar através dos da melhoria dos processos. \\
        \hline
        
        5. Orientação ao trabalho, tarefa ou processo & O principal propósito da TI é em transformar a direção estratégica da organização: ser mais focada nos negócios/consumidor, enquanto a mesma suporta as responsabilidades para chegar aos resultados com uma operação exemplar. \\ \hline
        
        6. Isolamento contra Colaboração/Cooperação & Cooperação, colaboração e entendimento mútuo entre a TI e o negócio são essenciais para o alinhamento. \\
        \hline
        
        7. Controle, coordenação e responsabilidade & O Conselho é o responsável pela implantação de um {\it framework} de governança de TI. O gerenciamento de TI e negócios tem a responsabilidade na implantação de controles adequados. Há benefícios em uma ação controlada e coordenada, com espaço para iniciativas e níveis de responsabilidades pessoais. \\
        \hline
        
        8. Orientação e foco - interno e/ou externo & A TI deve ser orientada pelo negócio, focada no consumidor e seus sistemas devem oferecer aporte a organização estratégica. A TI deve ter um papel de consultora de negócios e propagar as novidades tecnológicas. \\
        \hline
    \end{tabular}

\end{center}
\caption {{\bf Tabela 1: Modelo proposto de uma Cultura de governança em TI, segundo De Haes, Grembergen e Rowlands, 2007\cite{exploringit}}}

\\~\\

Porém o que tudo isto pode significar? Com tais conhecimentos sintetizados, adaptar a organização do ambiente de TI para alguma empresa que apresenta as características identificadas de acordo Detert et al, 2000, fica mais identificável e fácil de adaptar como a cultura devida de governança melhor pode se adequar a empresa em foco. Porém ainda assim, isto é algo que não é trivial, e mesmo com uma ferramenta que 

\section{Reflexões sobre o cenário}

Podemos concluir que as pessoas sabem como seu desempenho e/ou vontade de burlar as regras da empresa e, até mesmo, a falta de absorção da cultura organizacional, tem se voltado financeiramente para a empresa e, até mesmo, para o seu colaborador? Como será que o mundo está encarando estas mudanças? Como será que o Brasil tem encarado tais mudanças? Estamos enfrentando as mesmas dificuldades que boa parte do mundo tem enfrentado também com tal processo? Se estamos enfrentando, o que estamos fazendo para que as pessoas sejam bem sucedidas ao absorver e aplicar o conhecimento aprendido que a governança vem propor para nós? Se estamos sendo mal sucedidos na aplicação do conhecimento para o nosso colaborador e, até mesmo, usuário final (afinal, a governança em TI não se restringe apenas a negócios com fins lucrativos), o que estamos fazendo para atacar tal dificuldade? Como será que uma absorção mais eficiente poderia ajudar a empresa a estancar ou até mesmo melhorar